\makeglossaries

% Define glossary entries
\newglossaryentry{argument}{
    name=Argument,
    description={In Python, an argument is a value passed to a function or method when it is called. Arguments are specified after the function name inside the parentheses. They can be used in the function body as variables.}
}

\newglossaryentry{class}{
    name=Class,
    description={A class in Python is a blueprint for creating objects. Classes encapsulate data for the object and functions that manipulate that data.}
}

\newglossaryentry{clone}{
    name=Clone,
    description={Cloning in terms of a repository means creating a local copy of a remote repo. This is often done to create a local workspace where changes can be made independently of the original repository.}
}

\newglossaryentry{commit}{
    name=Commit,
    description={In version control systems, a commit is a snapshot of your entire repository at one point. It provides a record of what changes were made and who made them. Each commit includes a commit message that describes the changes.}
}

\newglossaryentry{decorator}{
    name=Decorator,
    description={A decorator in Python is a design pattern that allows a user to add new functionality to an existing object without modifying its structure. Decorators are usually called before the definition of a function you want to decorate.}
}

\newglossaryentry{dictionary}{
    name=Dictionary,
    description={A dictionary is a mutable, unordered collection of key-value pairs. Keys must be unique and immutable. Dictionaries allow for fast data lookup and are implemented using hash tables.}
}

\newglossaryentry{exception}{
    name=Exception,
    description={An exception is an event that occurs during a program's execution and disrupts the normal flow of its instructions. In Python, exceptions are triggered automatically on errors or can be triggered and intercepted by your code.}
}

\newglossaryentry{function}{
    name=Function,
    description={A function in Python is a block of organized, reusable code used to perform a single, related action. Functions provide better modularity for your application and a high degree of code reuse.}
}

\newglossaryentry{generator}{
    name=Generator,
    description={A generator in Python is a particular type of iterator that allows the user to return a value and resume processing to return subsequent values. Generators are written like regular functions but use the \texttt{yield} statement when they want to return data.}
}

\newglossaryentry{import}{
    name=Import,
    description={In Python, \texttt{import} is a keyword to import modules into the current namespace. Modules are Python .py files that consist of Python code.}
}

\newglossaryentry{iterable}{
    name=Iterable,
    description={An iterable in Python is any object capable of returning its members one at a time, permitting it to be iterated over in a loop. Common examples of iterables include lists, tuples, and dictionaries.}
}

\newglossaryentry{lambda}{
    name=Lambda,
    description={A lambda function is a small anonymous function defined with the \texttt{lambda} keyword. Lambda functions can have any number of arguments but only one expression.}
}

\newglossaryentry{list}{
    name=List,
    description={A list in Python is a mutable, ordered sequence of elements contained within square brackets. Lists are versatile and can be used to store a sequence of objects.}
}

\newglossaryentry{loop}{
    name=Loop,
    description={A control structure in Python that repeatedly executes a block of code as long as a given condition is true. Common types include loops, which iterate over a sequence, and loops, which continue based on a conditional test.}
}

\newglossaryentry{merge}{
    name=Merge,
    description={Merging is a version control operation that integrates changes from one repository branch into another. This can happen within a personal project or as a collaborative effort where changes from multiple contributors are combined.}
}

\newglossaryentry{module}{
    name=Module,
    description={A module is a file containing Python definitions and statements. The file name is the module name with the suffix \texttt{.py} added.}
}

\newglossaryentry{object}{
    name=Object,
    description={In Python, an object is an instance of a class. Objects have the data attributes and methods that were defined in the class.}
}

\newglossaryentry{package}{
    name=Package,
    description={A package is a namespace that contains multiple packages and modules. It is simply a directory with a Python file named \texttt{\_\_init\_\_.py}.}
}

\newglossaryentry{parameter}{
    name=Parameter,
    description={Parameters are variables that are defined in the function definition. They are used to receive and store arguments passed to a function at the call time.}
}

\newglossaryentry{pull}{
    name=Pull,
    description={A pull is a version control operation in which changes from a remote repository are downloaded and integrated into a local copy of a project. This is often used to synchronize a local repository with changes made by others.}
}

\newglossaryentry{push}{
    name=Push,
    description={A push is a version control operation to upload local repository changes to a remote repository. This operation allows other users to see the changes made when they pull from the repository.}
}

\newglossaryentry{repository}{
    name=Repository (Repo),
    description={A repository is a central file storage location used by version control systems to store multiple versions of files. In web development and software projects, a repository often refers to a storage location on services like GitHub or GitLab where code is stored, tracked, and managed.}
}

\newglossaryentry{tuple}{
    name=Tuple,
    description={A tuple is an ordered and immutable collection written with round brackets. It can contain a mix of objects.}
}

\newglossaryentry{variable}{
    name=Variable,
    description={A variable in Python is a reserved memory location for storing values. In other words, a variable in a Python program gives data to the computer for processing.}
}

\newglossaryentry{version}{
    name=Version,
    description={In software development, a version refers to a specific state of a software package or is saved in a version control system. Versions allow development involving multiple contributor versions if necessary.}
}

\newglossaryentry{virtual-environment}{
    name=Virtual Environment,
    description={A virtual environment is a self-contained directory tree that contains a Python installation for a particular version of Python, plus several additional packages.}
}